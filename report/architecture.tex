The autonomous vehicle system is divided into several interconnected modules that ensure efficient perception, localization and planning. 
The perception system includes cone detection, which is performed by a YOLO-based node that uses a trained deep learning model to identify cones in the camera image. This enables accurate detection of the route boundary. 
Lidar and camera data are continuously retrieved and used to perform perception. The sensor fusion module processes and combines data from these sources to create a comprehensive model of the environment.

The localization module is responsible for determining the exact position of the vehicle in relation to an HD map by using the fused data from the perception sensors. 
This module uses the SLAM toolbox for simultaneous localization and mapping, while RViz is used for real-time visualization of sensor data. It provides the position of the vehicle, which serves as essential input for path planning.

The path planning module is generates an optimal trajectory that ensures the vehicle navigates the track efficiently. It processes environmental information, vehicle dynamics, and track constraints to determine a feasible and optimized driving path.
The planning process consists of two stages. Initially, the system explores the track to gather key information about the environment and construct a representation of the drivable space. Once sufficient data is collected, the module transitions to a global planning mode, where it optimizes the trajectory based on the recorded driving path to ensure smooth navigation. 
This module operates in close coordination with the perception, localization, and control systems:
\begin{itemize}
    \item \textbf{Perception:} Provides real-time data on track boundaries.
    \item \textbf{Localization:} Ensures precise positioning of the vehicle relative to the planned path.
    \item \textbf{Control:} Executes the computed trajectory by adjusting steering and acceleration.
\end{itemize}