\textbf{TODO: Rework chapter and give more technical explanations.} During the development of the project the team faced multiple challenges. The core problem of the project was the translation of the car's behavior from the simulation into the real world. The following challenges required a lot of time to handle and overcome, which resulted in unexpected developmental delays.\\
\newline
Whilst working on the M2P node and working on the validation in the real world we had to find out, that ROS and Gazebo use different coordinate systems, which resulted in correct calculations in the simulations yet incorrect ones in real world testing. This was due to the odometry and the target points not adhering to the same coordinate system. This problem was overcome by writing a transform node which would translate between both coordinate systems depending on the usage context. 
Additionally, during a debugging session it occurred that there was another difference between simulation and real world. In the real world scenario, the racecar would stop just before the range for the M2P node to consider a point to be reached. This was due to a too small velocity in the drive message for the racecar to move forward. To overcome this issue, a minimum velocity was implemented and the handling of target points in the M2P node was adjusted slightly.\\
\newline
Furthermore the team encountered problems with the transform tree and working with the rosbags. These problems have expressed themselves in a heavily drifting odometry and lookup errors for sensor data. With these issues self-localization and mapping seemed impossible in a real world scenario. Fortunately, after a system update on the racecar the problem was solved and the racecar was driving in the ROS coordinate system.
