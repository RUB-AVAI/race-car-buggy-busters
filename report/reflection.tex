Looking back on the expectations to this project and the faced challenges during development, it is clear that not every requirement could be fulfilled with the given time constraints. While the racecar is able to execute basic autonomous driving tasks like cone detection and classification, self-localization, path planning, and map generation, some requirements that fall under the category ``could have'' were not realized.\\
\newline
These requirements were for example the emergency braking, failsafe mode, or the health monitoring. These were partially not implemented due to a lower priority. While the racecar does not register driving failures like leaving the track by itself, quitting the nodes does result in an immediate stop since the racecar does not receive any more drive messages. This resulted in the emergency stop to be obsolete for the current state. Due to the time constraints there is no guarantee on the quality requirements. \\
\newline
In further development, one could review the existing requirements - especially under the factor of actually achieved top speeds during the race - and adjust them. The optional requirements that were not fulfilled could be reviewed under the point on necessity and implemented given a prioritization. \\
\newline
\textbf{TODO: What could be changed in the development process, the tech stack, etc. for further improvement and for future iterations of the course?}