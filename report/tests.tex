During the development phase, we wanted to ensure that our steps worked as expected and delivered the right functionality. Two main approaches were taken during development. One was the use of Gazebo simulation to verify that specific implementations met our expectations. However, we were aware that the simulation does not necessarily have to match the behavior of the actual platform. For the car platform, we initially used pre-recorded rosbags from the F1TENTH system. This was one of the most efficient ways to test and analyze the vehicle’s response to real-world data.

In addition to simulation-based validation, we also implemented unit tests for several nodes during development to ensure the correctness of the functions within each node. These unit tests were designed to validate key components of the system, such as perception, control, and localization, by checking their outputs against expected values. By automating these tests, we were able to quickly detect and fix issues, ensuring the stability and reliability of the software. This approach provided an additional layer of confidence in our implementation before deploying the system to real-world scenarios.